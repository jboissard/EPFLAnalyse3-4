\chapter[Fonctions complexes]{Fonctions complexes}

\section{Généralités}
$i^2=\sqrt{-1}$

Tout d'abord on a 


En coordonnées cartésiennes:
\begin{myDefinition}
$z = x + iy $
\\
$f(z)=Re f(z) + iIm f(z) = u(x) + iv(x)$
\end{myDefinition}


où $x$ est la partie réelle du nombre complexe $Re(z)$ et $y$ la partie imaginaire de $z$ $Im(z)$. 
x,y  $\in $ $\R$ et $z$ définit un nouvel ensemble $\C$, l'ensemble des complexes.

Module de z:
$|z| = \sqrt{x^2+y^2} = 2$

Argument:
$\theta = arg(z) + 2k\pi$

Conjugué complexe:
$\overline{z}=x-iy$ 
et
$z\overline{z}=|z|^2=x^2 + y^2$

si $z \neq 0$
$z^{-1} = \frac{\overline{z}}{{\left| z \right|}^2}$

En coordonnées polaires:
$z = re^{i\theta} = |z|e^{iarg(z)}$

$e^{iz}=\cos{z} + i\sin{z}$

Identité d'Euler:
$e^{i\pi} -1 = 0 $

\begin{myTheorem}
Fromule de De Moivre
$\left( e^{iz} \right)^n =e^{inz}=\cos{nz} + i\sin{nz}$
\end{myTheorem}

si $z_{1},z_{2} \neq 0$
$z_{1}=z_{2} \Leftrightarrow |z_{1}|=|z_{2}|$ et $arg(z_{1})=arg(z_{2})$

Mettre image plan z équivalent à plan $\R^2$

\begin{myExample}
	fonction exponentielle
	$z = x + iy$
	\\
	$e^z=e^xe^{iy} = e^x(\cos{y} + i\sin{y})$
	\\
	$u(x,y)=e^x\cos{y}$
	\\
	$v(x,y)=e^x\sin{y}$
	\begin{itemize}
		\item $e^z$ est $2\pi i$ périodique
		\item $e^{z_{1}}e^{z_{2}}=e^{z_{1}+z_{2}}$
	\end{itemize}
\end{myExample}



Quelques fonctions utiles:
\begin{itemize}
	\item $\cos{z} = Re\{e^{iz}\} = \frac{e^{iz}+e^{-iz}}{2}$
	\item $\sin{z} = Im\{e^{iz}\}= \frac{e^{iz}-e^{-iz}}{2i}$
	
	\item $\cosh{z} = \frac{e^{z}+e^{-z}}{2}$
	\item $\sinh{z} = \frac{e^{z}-e^{-z}}{2}$
	
\end{itemize}

$\Rightarrow$ Toutes continues
	
\section{Fonction logarithme}
\begin{myDefinition}
	$\ln(z) = \ln(|z|) + i \left ( arg(z) + 2 \pi k \right )$ $z \in \C^*$
\end{myDefinition}

\begin{myProperty}
	\begin{eqnarray*}
		e^{ln(z)} = e^{ln(|z|)}e^{iarg(z)} = |z|e^{iarg(z)} = z
		\\
		ln(e^z)=ln|e^z| + iarg(e^z) = x + iy = z
		\\
		ln(ab)=ln(a) + ln(b)
	\end{eqnarray*}
\end{myProperty}

$ln(z) =ln|z| + iarg(z)$
\\
$u(x,y)=ln|z| = ln(x^2+y^2)$
\\
$v(x,y)= arg(z) = \left\{\begin{matrix}  
\arctan \frac{y}{x} &\mbox{if}\ x > 0, y \in \R 
\\
\frac{\pi}{2} &\mbox{if}\ x = 0, y>0 
\\
-\frac{\pi}{2} &\mbox{if}\ x = 0, y<0 
\\
\pi+\arctan \frac{y}{x} &\mbox{if}\ x <0, y>0
\\
-\pi+\arctan \frac{y}{x} &\mbox{if}\ x<0, y<0
\end{matrix}\right.$
\\
et donc
\\

$v(x,y) = \left\{\begin{matrix}  
\arctan \frac{y}{x} &\mbox{if}\ x > 0, y \in \R 
\\ 
\frac{\pi}{2}-\arctan \frac{x}{y} &\mbox{if}\ x \in \R, y>0
\\
\frac{\pi}{2}+\arctan \frac{x}{y} &\mbox{if}\ x \in \R, y<0
\end{matrix}\right.$

%\begin{myQuote}{Johan \textsc{Boissard}, Auteur de cet aide-mémoire, Hiver 2008}
%Il y a ceux qui regardent la télé et ceux qui la font.
%\end{myQuote}

\begin{myExample}
Résoudre $(z^4-1)sin(\pi z)=0$
\\
$z^4-1=0$
\\
avec $y = z^2$
\\
$y^2 -1=(y-1)(y+1)=0$
\\
$y=z^2=\pm 1$ et donc $z^2-1=(z-1)(z+1)=0$ ce qui donne $z=\pm 1$
\\
et $z^2+1=(z-i)(z+i)=0$ et donc $z=\pm i$
\\
mais on a aussi $sin(\pi z)=0$
\\
$sin(z)$ s'annule tous les multiples de $\pi$ donc $z=k$ avec $k \in \Z$
\\
finalement $z=\left\{\begin{matrix} k &\mbox{for}\ k \in \Z \\i\\-i\end{matrix}\right.$
\end{myExample}









