\chapter[Théorème de Stokes]{Théorème de Stokes}

\section{Généralités}

\begin{myTheorem} {\bf Théorème de Stokes}
	\\Soit $\Sigma\subset\R^3$ une surface régulière par morceaux et orientable. Soit $F:\Sigma\rightarrow\R^3,F=(F_1,F_2,F_3)$, où les $F_i$ sont $C^1$ sur un ouvert contenant $\Sigma\cup\partial\Sigma$, alors
	\begin{eqnarray*}
		\iint_\Sigma \mathrm{rot}F\cdot ds=\int_{\partial\Sigma} F\cdot dl
	\end{eqnarray*}
	{\bf Remarques}
	\begin{enumerate}
		\item pour toutes les surfaces régulières par morceaux $\Sigma \subset\R^3$, avec paramétrisation $\sigma=\sigma(u,v):\overline{A}\rightarrow\Sigma$, que nous considérons, le bord de $\Sigma$, noté $\partial\Sigma$, est donné par $\sigma(\partial A)$ où on a enlevé les parties qui sont parcourues deux foisdans des sens opposés ainsi que les points
		\item Une fois la paramétrisation choisie $\sigma:\overline{A}\rightarrow\Sigma$, la normale $\sigma_u\wedge\sigma_v$, c'est-à-dire
		\begin{eqnarray*}
			\iint_\Sigma \mathrm{rot}F\cdot ds=\iint_A (\mathrm{rot}F(\sigma(u,v))\cdot\sigma_u\wedge\sigma_v)\,dudv
		\end{eqnarray*}
		Par ailleurs le sens de parcours de $\partial\Sigma$ est alors celui induit par la paramétrisation $\sigma:\overline{A}\rightarrow\Sigma$ et c'est donc celui obtenu en parcourant positivement $\partial A$
	\end{enumerate}
\end{myTheorem}

\begin{myExample} Vérifier le théorème de Stokes pour $F(x,y,z)=(z,y,x)$ et
	\begin{eqnarray*}
		\Sigma=\{(x,y,z)\in\R^3:z^2=x^2+y^2\,\,et\,\,0<z<1\}
	\end{eqnarray*}
	\begin{enumerate}
		\item 
			Calcul de $\iint_\Sigma\mathrm{rot}F\cdot ds$
			\\
			on a
			\begin{eqnarray*}
				\nabla\times F = 
				\begin{vmatrix}
					e_1&e_2&e_3
					\\
					\frac{\partial}{\partial x}&\frac{\partial}{\partial y}&\frac{\partial}{\partial z}
					\\
					z&x&y
				\end{vmatrix}=
				(1,1,1)
			\end{eqnarray*}
			Pour la paramétrisation, on a
			\begin{eqnarray*}
				\sigma(\theta,z)=(z\cos\theta,z\sin\theta,z)\,\,
				avec
				\,\,
				(\theta,z)\in A=(0,2\pi)\times(0,1)
			\end{eqnarray*}
			la normale est donnée par
			\begin{eqnarray*}
				\sigma_\theta\wedge\sigma_z=
				\begin{vmatrix}
					e_1&e_2&e_3
					\\
					-z\sin\theta&z\cos\theta&z
					\\
					\cos\theta&\sin\theta&1
				\end{vmatrix}=
				(z\cos\theta,z\sin\theta,-z)
			\end{eqnarray*}
			On peut maintenant passer au calcul de l'intégrale
			\begin{eqnarray*}
				\iint_\Sigma\mathrm{rot}F\cdot ds=
				\\
				\int_{0}^{2\pi}\int_{0}^{1}(1,1,1)\cdot(z\cos\theta,z\sin\theta,-z)\,dzd\theta
				\\
				\int_{0}^{2\pi}\int_{0}^{1}z\cos\theta+z\sin\theta-z\,dzd\theta
				\\
				\int_{0}^{1}z\,dz\int_{0}^{2\pi}\cos\theta+\sin\theta-1\,d\theta
				\\
				\frac{1}{2}\cdot(-2\pi)=
				\\
				-\pi			
			\end{eqnarray*}
		\item
			Calcul de $\int_{\partial\Sigma}F\cdot dl$
			\\
			Commençons par calculer
			\begin{eqnarray*}
				\sigma(\partial A)=\Gamma_1\cup\Gamma_2\cup\Gamma_3\cup\Gamma_4
			\end{eqnarray*}
			où
			\begin{eqnarray*}
				\Gamma_1=\{\gamma_1(\theta)=\sigma(\theta,0)=(0,0,0)\}={(0,0,0)}
				\\
				\Gamma_2=\{\gamma_2(z)=\sigma(2\pi,z)=(z,0,z),z:0\rightarrow 1\}
				\\
				\Gamma_3=\{\gamma_3(\theta)=\sigma(\theta,1)=(\cos\theta,\sin\theta,1),\theta:2\pi\rightarrow 0\}
				\\
				\Gamma_4=\{\gamma_4(z)=\sigma(0,z)=(z,0,z),z:1\rightarrow 0\}=-\Gamma_2
			\end{eqnarray*}
			On remarque que
			\begin{eqnarray*}
				\partial\Sigma=\Gamma_3
			\end{eqnarray*}
			et on a $\gamma_3'(\theta)=(-\sin\theta,\cos\theta,0)$
			On trouve donc
			\begin{eqnarray*}
				\int_{\partial\Sigma}F\cdot dl=
				\\
				\int_{2\pi}^{0}(1,\cos\theta,\sin\theta)\cdot(-\sin\theta,\cos\theta,0)\,d\theta=
				\\
				\int_{2\pi}^{0}-\sin\theta+\cos^2\theta\,d\theta=
				\\
				\left[\frac{1}{2}\theta+\frac{\sin2\theta}{4}\right]_{2\pi}^0=
				\\
				-\pi
			\end{eqnarray*}
	\end{enumerate}
	On a bien vérifié le théorème de Stokes dans ce cas.
\end{myExample}

\begin{myExample}
Vérifier le théorème de Stokes pour $F(x,y,z)=(0,0,y^2)$ et
	\begin{eqnarray*}
		\Sigma=\{(x,y,z)\in\R^3:x^2+y^2+z^2=1\,\,et\,\,z\leq 0\}
	\end{eqnarray*}
	\begin{enumerate}
		\item 
			Calcul de $\iint_\Sigma\mathrm{rot}F\cdot ds$
			\\
			on a
			\begin{eqnarray*}
				\nabla\times F = 
				\begin{vmatrix}
					e_1&e_2&e_3
					\\
					\frac{\partial}{\partial x}&\frac{\partial}{\partial y}&\frac{\partial}{\partial z}
					\\
					0&0&y^2
				\end{vmatrix}=
				(2y,0,0)
			\end{eqnarray*}
			On paramétrise de la façon suivante
			\begin{eqnarray*}
				\sigma(\theta,\phi)=(\cos\theta\sin\phi,\sin\theta\sin\phi,\cos\phi)
			\end{eqnarray*}
			avec $(\theta,\phi)\in A=(0,2\pi)\times(\frac{\pi}{2},\pi)$
			\\On a la normale suivante
			\begin{eqnarray*}
				\sigma_\theta\wedge\sigma_\phi=
				\begin{vmatrix}
					e_1&e_2&e_3
					\\
					-\sin\theta\sin\phi&\cos\theta\sin\phi&0
					\\
					\cos\theta\cos\phi&\sin\theta\cos\phi&-\sin\phi
				\end{vmatrix}=
				\\
				-\sin\phi\,\sigma(\theta,\phi)
			\end{eqnarray*}
			On peut maintenant calculer l'intégrale
			\begin{eqnarray*}
				\iint_\Sigma\mathrm{rot}F\cdot ds=
				\\
				\int_{\frac{\pi}{2}}^\pi\int_{0}^{2\pi}(2\sin\theta\sin\phi,0,0)\cdot-\sin\phi\,\sigma(\theta,\phi)\,d\theta d\phi=
				\\
				\int_{\frac{\pi}{2}}^\pi\int_{0}^{2\pi}-2\sin\theta\cos\theta\sin^3\phi\,d\theta d\phi=
				\\
				-2\int_{\frac{\pi}{2}}^\pi\sin^3\phi\,d\phi\int_{0}^{2\pi}\sin\theta\cos\theta\,d\theta=
				\\
				-2\int_{\frac{\pi}{2}}^\pi\sin^3\phi\,d\phi\left[\frac{\sin^2\theta}{2}\right]_{0}^{2\pi}=
				\\
				0
			\end{eqnarray*}
		\item
			Calcul de $\int_{\partial\Sigma}F\cdot dl$
			\\
			Commençons par calculer
			\begin{eqnarray*}
				\sigma(\partial A)=\Gamma_1\cup\Gamma_2\cup\Gamma_3\cup\Gamma_4
			\end{eqnarray*}
			où
			\begin{eqnarray*}
				\Gamma_1=\{\gamma_1(\theta)=\sigma(\theta,\frac{\pi}{2})=(\cos\theta,\sin\theta,0),\theta:0\rightarrow 2\pi\}
				\\
				\Gamma_2=\{\gamma_2(\phi)=\sigma(2\pi,\phi)=(\sin\phi,0,\cos\phi),\phi:\frac{\pi}{2}\rightarrow \pi\}
				\\
				\Gamma_3=\{\gamma_3(\theta)=\sigma(\theta,\pi)=(0,0,-1)\}=\{(0,0,-1)\}
				\\
				\Gamma_4=\{\gamma_4(\phi)=\sigma(0,\phi)=(\sin\phi,0,\cos\phi),\phi:\pi \rightarrow \frac{\pi}{2}\}=-\Gamma_2
			\end{eqnarray*}
			On remarque que
			\begin{eqnarray*}
				\partial\Sigma=\Gamma_1
			\end{eqnarray*}
			et on a $\gamma_3'(\theta)=(-\sin\theta,\cos\theta,0)$
			On trouve donc
			\begin{eqnarray*}
				\int_{\partial\Sigma}F\cdot dl=
				\\
				\int_{0}^{2\pi}(0,0,\sin^2\theta)\cdot(-\sin\theta,\cos\theta,0)\,d\theta=
				\\
				\int_{0}^{2\pi}\sin^2\theta\cdot 0\,d\theta=
				\\
				0
			\end{eqnarray*}
	\end{enumerate}
\end{myExample}