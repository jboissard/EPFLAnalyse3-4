\chapter{Les distributions tempérées}
\section{Introduction}
\begin{itemize}
	\item O. Heaviside 1894
	\item P. Dirac 1925
	\item L. Schwartz 1945
\end{itemize}
\underline{Une motivation:} décrire mathématiquement un \underline{choc} (impulsion forte de courte durée).

\underline{Exemple:} Objet de masse $1$, mis en mouvement par un coup de marteau.
\begin{eqnarray}
	v(t)=\text{vitesse de la masse à l'instant } t
	\\
	1\cdot v'(t)=\text{force}=f_\epsilon(t)=\begin{cases}
		0&\text{ si }t<0
		\\
		\frac{1}{\epsilon}&\text{ si }0<t<\epsilon
		\\
		0&\text{ si }t>\epsilon
	\end{cases}
\end{eqnarray}


%todo deux graph ici

Une \underline{impulsion} serait un objet mathématique $\delta$ tel que la solution de $y'(t)=\delta(t)$ $t\in \mathbb R$ est
\begin{eqnarray}
	y(t)=
	\begin{cases}
		0&\text{ si } t<0
		\\
		1&\text{ si } t>0
	\end{cases}
\end{eqnarray}
%graph ici
Il n'ya pas de fonction avec cette propriété!
\section{Interactions entre fonctions}
\subsection{Concept classique}
$f:\mathbb R\rightarrow\mathbb R$ est déterminé par les valeurs $f(x)$ $x\in\mathbb R$
\subsection{Concept nouveau}
les fonctions interragissent entre elles par l'intermédiare du produit scalaire
\begin{eqnarray}
	<f,g>=\int_{-\infty}^{\infty}f(x)g(x)dx
\end{eqnarray}
\subsubsection{Ces deux concepts sont proches}
Si $f$ est continue et intégrable, alors les valeurs de $f$ sont déterminés par la donné de
\begin{eqnarray*}
	<f,g>, \,g\text{ bornée}
\end{eqnarray*}
Problème dans la définition de $<f,g>$: l'intégrale ne converge pas forcément.

On va se restreindre à 2 classes de fonctions, une pour $f$, l'autre pour $g$
\subsection{L'espace $\mathop S$ des fonctions à décroissance rapide (espace de Schwartz)}
\underline{Définition:} $\mathcal S$ est l'espace vectoriel des fonctions $\varphi\in C^{\infty}$ de $\mathbb R$ dans $\begin{cases}
	\mathbb R\\
	\mathbb C
\end{cases}$ telles que, pour tout $n\in\mathbb N$ et $m\in\mathbb N$,
\begin{eqnarray}
	\lim_{x\rightarrow \pm\infty} x^n\varphi^{m}(x)
\end{eqnarray}
cette condition signifie que $\varphi$ et toutes ses dérivées tendent vers $0$ très rapidement (plus rapidement que $\frac{1}{x^n}$ avec $n\in\mathbb N$)
\subsubsection{Exemples}
\begin{enumerate}
	\item $\varphi(x)=e^{-x^2}$, alors $\lim_{x\rightarrow \pm\infty}x^ne^{-x^2}=0$
	
	
	\item $\Phi=
	\begin{cases}
		0&\text{si }x\leq0
		\\
		e^{-\frac{1}{x}}&\text{si }x>0
	\end{cases}$
	
	\item Soit $a<b$, posons $p(x)=\Phi(x-a)\Phi(x-b)=
	\begin{cases}
		\exp{(-\frac{b-a}{(x-a)(x-b)})}&\text{si }x\in[a,b]
		\\
		0&\text{sinon }%x\leq a
	\end{cases}$
\end{enumerate}

\subsection{Fonctions à croissance lente continue (CCL)}
\underline{Définition:} (a) une fonction CCL est une fonction
$
	f:\mathbb R\rightarrow
	\begin{cases}
		\mathbb R
		\\
		\mathbb C
	\end{cases}$ telle qu'il existe $n\in\mathbb N$ tel que
	\begin{eqnarray*}
		\lim_{x\rightarrow\pm\infty}\frac{f(x)}{x^n}=0
	\end{eqnarray*}
	(pour un certain $n\in\mathbb N$, $f(x)$ grandit moins vite que $x^n$, lorsque $x\rightarrow\pm\infty$)
	
\subsubsection{Exemples}
\begin{enumerate}
	\item $x^4+x^2+x$\\
	$\sin{x},\cos(x)$\\
	$\ln{(|x|)}$ sont CCL
	
	\item $f(x)=e^x$ n'est \underline{pas} CCL, car $\lim_{x\rightarrow\infty}=\infty$
\end{enumerate}
\section{Les distributions tempérées}
\underline{Définition: } (b) une fonctionelle sur $\mathcal S$ est une application linéaire $T: \varphi\rightarrow
\begin{cases}
	\mathbb R
	\\
	\mathbb C
\end{cases}$
définie par
\begin{eqnarray}
	\label{eq:deffourierchap6fccl}
	<T_f,\varphi>=\int_{-\infty}^{\infty}f(x)\varphi(x)dx&\varphi\in\mathcal S
\end{eqnarray}
Cette intégrale (eq:\ref{eq:deffourierchap6fccl}) est toujours bien définie.

\underline{Définiton:} On appelle $T_f$ la \underline{fonctionelle associée} à $f$, donc on écrit
\begin{eqnarray}
	<f,\varphi>=\int_{-\infty}^{\infty}f(x)\varphi(x)dx
\end{eqnarray}

\underline{Définition: }
une distribution tempérée (ou distribution de Schwartz) est fonctionelle $T_f^{(n)}:\varphi\rightarrow
\begin{cases}
	\mathbb R
	\\
	\mathbb C
\end{cases}$
de la forme
\begin{eqnarray}
	<T_f^{(n)},\varphi>=(-1)^n\int_{-\infty}^{\infty}f(x)\varphi^{(n)}(x)dx
\end{eqnarray}
où $\varphi^n(x)$ est la $n^{\text{ème}}$ dérivée de $\varphi$, $n\in\mathbb N$
\\
où $f$ est CCL.

L'ensemble des distributions tempérées est noté $\mathcal S'$. 

\underline{Remarques:}
$T_f^{(0)}=T_f$: $T_f^{(0)}$ est identifiée à $f$.

\underline{Proposition:} (lien entre $T_f^{(1)}$ et $f$)

Supposons que $f'$ est aussi CCL, Alors $T_f^{(1)}=T_{f'}$, $T_f^{(1)}$ peut être identifiée à $f'$.
%il y a une démonstration ici

\underline{Interprétation}
\begin{enumerate}
	\item $T_f^{(1)}$ est appelée la \underline{dérivée} (au sens des distributions) de $T_f$ et même de $f$.
	\item Si $f\in C^{(n)}$ et $f',f'',f^{(n)}$ CCL, alors
	\begin{eqnarray*}
		<T_f^{(n)},\varphi'>=<f^{(n)},\varphi>=\int_{-\infty}^{\infty}f^{(n)}\varphi(x)dx
	\end{eqnarray*}
	\item On écrit $(T_f)'=T_f^{(1)}=f'$, $(T_f)''=T_f^{(2)}=(T_f')'=f''$, étant entendu que dans ce cas, $f_1,f_2,\dots,\text{etc...}$ représentent des distributions.
\end{enumerate}

\subsection{La distribution $\mathcal S$}

\underline{Définition:}

Soit $\varphi\rightarrow\in R$ définie par $<\delta,\varphi>=\varphi(0)$, $\varphi\in\mathcal S$

On appelle $\delta$ la \underline{fonctionelle de Dirac}.

\underline{Proposition:} 

Posons $r(x)=
\begin{cases}
	0&\text{si }x<0
	\\
	x&\text{si }x\geq0
\end{cases}$
%mettre plot ici

$H(x)=
\begin{cases}
	0&x<0
	\\
	1&x\geq0
\end{cases}$

Alors $r'=H$ et $r''=H'=\delta$, c'est-à-dire
\begin{eqnarray*}
	\int_{-\infty}^{\infty}r(x)\varphi'(x)dx=\int_{-\infty}^{\infty}H(x)\varphi(x)dx
\end{eqnarray*}
et
\begin{eqnarray*}
	(-1)^2\int_{-\infty}^{\infty}r(x)\varphi''(x)dx=\int_{-\infty}^{\infty}H(x)\varphi'(x)dx=<\delta,\varphi>=\varphi(0)
\end{eqnarray*}

Démonstration: cf exercices

\underline{Remarques:}

$H'=\delta$
\begin{eqnarray*}
	H(t)=
	\begin{cases}
		0&\text{si }t<0
		\\
		1&\text{ si }t\geq0
	\end{cases}
\end{eqnarray*}
$\delta$ est l'objet mathématique qui correspond à la notion d'impulsion.

(b) On peut montrer que $\delta$ est la limite des fonctions $f_\epsilon$ dans l'exemple de la première heure.

\section{Transformée de Fourier d'une distribution tempérée}

\underline{Définition:}

Soit $T\in\mathcal S'$. La transformée de Fourier de $T$ est l'élément $\mathcal F_T$ de $\varphi'$ défini par $<\mathcal F_T,\varphi>=<T,\mathcal F_\varphi>$, où $\varphi\in\mathcal S$.

\underline{Remarques:}
\begin{enumerate}
	\item On peut montrer que si $\varphi\in\mathcal S$, alors $\mathcal F_\varphi\in\mathcal S$ (où $\mathcal F_\varphi=\frac{1}{\sqrt{2\pi}}\int_{-\infty}^{\infty}\varphi(x)e^{-i\alpha x}$).
	\item Si $f:\mathbb R\rightarrow\mathbb R$ est telle que
	\begin{eqnarray*}
		\int_{-\infty}^\infty|f(x)|dx<+\infty
	\end{eqnarray*}
	ce qui signifie que $\mathcal F f$ est défini comme au sens du chapitre \ref{chap:FourierTrans}, alors
	\begin{eqnarray}
		<\mathcal F f,\varphi>=<f,\mathcal F\varphi> 
	\end{eqnarray}
\end{enumerate}
de sorte que $T_{\mathcal F_f}=\mathcal F(T_f)$.

\textbf{Proposition} Pour $a\in\mathbb R$, définissons $\delta_a:\mathcal S\rightarrow\mathbb R$ par
\begin{eqnarray}
	<\delta_a,\varphi>=\varphi(a)
\end{eqnarray}
(en particulier, $\delta_0$ est la distribution de Dirac $\delta$). Alors:
\begin{enumerate}
	\item $\mathcal F\delta_a$ est la distribution associée à la fonction CCL $x\mapsto\frac{1}{\sqrt{2\pi}}e^{-iax}$. 
En particulier, $\mathcal F\delta$ est la fonction constante égale à $\frac{1}{\sqrt{2\pi}}$: $\mathcal F\delta=\frac{1}{\sqrt{2\pi}}$;
\item $\mathcal F(e^{iax})=\sqrt{2\pi}\delta_a$;
\item $\mathcal F (\cos (x))=\sqrt{\frac{\pi}{2}}(\delta_1+\delta_{-1})$
\item $\mathcal F (\sin (x))=i\sqrt{\frac{\pi}{2}}(\delta_{-1}-\delta_{1})$
\end{enumerate}

\section{Opérations sur les distributions tempérées}
\begin{myDefinition}
	\textbf{Réflection}
	Etant donné une fonction $f:\mathbb R\rightarrow\mathbb R$, on définit une  nouvelle fonction $f^v:\mathbb R\rightarrow\mathbb R$ par $f^v(x)=f(-x)$ ($f^v$ se lit "$f$ check").

Pour $T\in\mathcal S'$, on définit une nouvelle distribution tempérée $T^v\in\mathcal S'$ par
\begin{eqnarray}
	<T^v,\varphi>=<T,\varphi^v> 
\end{eqnarray}
%voir polycop pour demonstration
\end{myDefinition}
\begin{myDefinition}
	\textbf{Translation}
	Etant donné une fonction $f:\mathbb R\rightarrow\mathbb R$, on définit une  nouvelle fonction $\mathcal T_a f:\mathbb R\rightarrow\mathbb R$ par $\mathcal T_af(x)=f(x+a)$
	
Pour $T\in\mathcal S'$, on définit une nouvelle distribution tempérée $\mathcal T_aT\in\mathcal S$ par
\begin{eqnarray}
	<\mathcal T_aT,\varphi>=<T,\mathcal T_{-a}\varphi> 
\end{eqnarray}
\end{myDefinition}

\begin{myDefinition}
	\textbf{Changement d'échelle}
	Etant donnés $a\neq0$ et une fonction $f:\mathbb R\rightarrow\mathbb R$, on définit une nouvelle fonction $\mathcal S_af(x)=f(ax)$.
	
Pour $T\in\mathcal S'$, on définit une nouvelle distribution tempérée $\mathcal S_aT\in\mathcal T'$ par
\begin{eqnarray}
	<\mathcal S_aT,\varphi>=<T,\frac{1}{|a|}\mathcal S_{1/a}\varphi>
\end{eqnarray}
\end{myDefinition}

\begin{myDefinition}
	Multiplication par une fonction $C^\infty CL$ 
	Une fonction $f:\mathbb R\rightarrow\mathbb C$ est dite $C^\infty$ \emph{à croissance lente} si $f$ est $C^\infty$.
	
Nous allons définir le produit d'une fonction $C^\infty CL g$ avec une distribution $T\in\mathcal S'$: on définit $g\cdot T\in\mathcal S'$ par
\begin{eqnarray}
	<g\cdot T,\varphi>=<T,g\cdot \varphi>.
\end{eqnarray}
\end{myDefinition}


\begin{myProposition}
	Avec les définitions ci-dessus, la transformée de Fourier possède les propriétés suivantes:
	\begin{enumerate}
		\item (Linéarité) Pour $a,b\in\mathbb R$ et $T_1,T_2\in\mathbb S'$,
		\begin{eqnarray}
			\mathcal F(aT_1+bT_2)=a\mathcal F(T_1)+b\mathcal F(T_2)
		\end{eqnarray}
		\item (Transformée de Fourier de dérivées)
		Soit $T\in\mathbb S'$ et $n\in\mathbb N$. Alors
		\begin{eqnarray}
			\mathcal F(T^{(n)})=(ix)^n\mathcal FT 
		\end{eqnarray}
		\item (Transformé de Fourier de la translatée) Soit $T\in\mathbb S'$ et $a\in\mathbb R$. Alors
		\begin{eqnarray}
			\mathcal F(\mathcal T_aT)=e^{iax}\mathcal FT 
		\end{eqnarray}
		ce qui signifie que 
		\begin{eqnarray}
			<\mathcal T_aT,\mathcal F\varphi>=<e^{iax}\mathcal FT,\varphi>=<\mathcal FT,e^{iax}\varphi>=<T,\mathcal F(e^{iax}\varphi)> 
		\end{eqnarray}
		\item (Transformée de Fourier et changement d'échelle)
		Soit $T\in\mathbb S'$ et $a\neq0$. Alors
		\begin{eqnarray}
			\mathcal F(S_aT)=\frac{1}{|a|}S_{1/a}(\mathcal FT)
		\end{eqnarray}
		
		\item (Dérivée de la transformée de Fourier)
		Soit $T\in\mathbb S'$. Alors
		\begin{eqnarray}
			(\mathcal FT)^{(n)}=(-i)^n\mathcal F(x^nT)
		\end{eqnarray}
	\end{enumerate}
\end{myProposition}

\section{Convolution des distributions tempérées}
L'objectif ici est de définir le produit de convolution $T_1*T_2$ de deux distributions tempérées $T_1,T_2\in\mathcal S'$.

La proposition suivante présente deux observations concernant des propriétés de la convolution de fonctions.
\begin{myProposition}
	\begin{enumerate}
		\item Si $f:\mathbb R\rightarrow\mathbb R$ est CCL et $\varphi\in\mathcal S$, alors 
		\begin{eqnarray}
			(f*\varphi)(x)=<f,\mathcal T_{-x}(\varphi^v)>
		\end{eqnarray}
		\item Soit $f:\mathbb R\rightarrow\mathbb R$ intégrable et $g:\mathbb R\rightarrow\mathbb R$ bornée  et $\varphi\in\mathcal S$. Alors
		\begin{eqnarray}
			\int_{-\infty}^\infty(g*f)(x)\varphi(x)\,dx=\int_{-\infty}^\infty f(x)(g*\varphi^V)(x)\,dx 
		\end{eqnarray}
	\end{enumerate}
\end{myProposition} 
%todo a finir

\section{Equations différentielles avec des distributions}
\end{document}
